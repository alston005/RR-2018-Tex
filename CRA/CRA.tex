  % !TEX TS?program = pdflatexmk
                    \documentclass[a4paper, 12pt]{article}
                  \usepackage[english]{babel}
                  \usepackage[utf8]{inputenc}
    		\usepackage[utf8]{inputenc}
    %                \renewcommand{\baselinestretch}{1.0} 
                    
                    
                    %MARGINS
                    \usepackage[left=25.4mm, right = 25.4mm, top=25.4mm, bottom=25.4mm, includefoot]{geometry}
      %            \geometry{a4paper, total={170mm,257mm}, left=25.4mm, right = 25.4mm, top=25.4mm, bottom=25.4mm}
                    \setlength{\parindent}{0in}
                    \usepackage{enumitem}
                   
                    %Adding pictures
                    \usepackage{graphicx}
                    \usepackage{float}
                    
                    %Header and footers
                    \usepackage{fancyhdr}
                    \pagestyle{fancy}
%                    \fancyhead{}
                    \fancyfoot{}
                    \fancyfoot[R]{ \thepage\ }
                    \renewcommand{\headrulewidth}{0pt} %change the pt width to insert header line
                    \renewcommand{\footrulewidth}{0pt} %change the pt width to insert footer line
                    \usepackage{amsmath}
                    \fancyhf{}
                    
%    	\rhead{\leftmark}
%    	\lhead{Guides and tutorials}
	\rfoot{Centre for Civil Society \hspace{1mm} \textbar \hspace{1mm} www.ccs.in \hspace{2mm}  \thepage}
%             \lfoot{  \leftmark }     			% get the section heading on the footer 
                   
                   
                    %Tables
                    \usepackage{booktabs}
                    \usepackage{subfig}
                    \captionsetup{aboveskip=14pt,}
                    \captionsetup[table]{singlelinecheck = false}
                    \newcommand{\tabitem}{~~\llap{\textbullet}~~} %new
                    
                    %Coloured Boxes
                    \usepackage{xcolor}
                    \usepackage{mdframed}
                    
                    %Custom Spacing
                    \usepackage{setspace}
                    
                    %Defining Colours
             \definecolor{CCSbrown}{RGB}{163, 86, 37}
               \definecolor{CCSgrey}{RGB}{105, 105, 105}
                 \definecolor{CCSblack}{RGB}{64, 64, 65} 
             
             
             %Heading colours                  
                    \usepackage{sectsty}
                    \usepackage{titlesec}
    		\chapterfont{\color{blue}}  % sets colour of chapters font
    		\sectionfont{\color{CCSbrown}}  %sets colour of section font
    		\subsectionfont{\color{CCSblack}} %sets colour of subsection font
    		\subsubsectionfont{\color{CCSgrey}} %sets colour of subsubsection font
    		
		
		%Bibliography
		
		\usepackage[authordate, backend=biber]{biblatex-chicago}
		\addbibresource{ewaste_v2.bib}
		\usepackage{hyperref} %[hidelinks]?
		\hypersetup{
		colorlinks,
		linkcolor = black,
		citecolor = blue}
		\usepackage{blindtext}
		
              %Tables 
              \usepackage{array}
              \usepackage{makecell}
              \usepackage{tabularx}	
              \newcolumntype{$}{>{\global\let\currentrowstyle\relax}}
              \newcolumntype{^}{>{\currentrowstyle}}
              \newcommand{\rowstyle}[1]{\gdef\currentrowstyle{#1}#1\ignorespaces}
              
                    \begin{document}
                    
    %==================================================                
                    %TITLE PAGE
                    \begin{titlepage}
                    	\begin{center}
                    	\line(1,0){300}\\
                    	[0.25in]
                    	\huge{\bfseries \textcolor{CCSbrown} {Risky Business}} \\
    	[0.5cm]
    	\large  {Assessing the Pollution Monitoring and Enforcement Framework\\  for Enterprises in Delhi} \\
    	
                    	\line(1,0){200}\\
                    	[1in]
                    	\textsc{\huge Ayesha Selwyn, \\ Parth Singh, Pushyami Chilakapati} \\
                    	[1.5cm]
                    	{\Large September 2018} \\
                    	[2.0cm]
    %    	{\huge Researching Reality Internship 2018} \\
     %   	[0.5cm]
                    	{\LARGE Centre for Civil Society} \\
                    	[0.1mm]
                    	{\Large New Delhi, India} \\
    	[2.0cm]
    	\includegraphics[width = 75mm]{/Users/pushyami/Desktop/LatexR/RR Papers/RR-2018-Tex/CRA/unnamed.png}
      
                    	\end{center}
                    \end{titlepage}
                     %=====================TOC===============================================                 
                    \tableofcontents
                    
     %======================LIST OF ABBREVIATIONS================================         
                   \newpage
                   \newlist{abbrv}{itemize}{1}
        \setlist[abbrv,1]{label=,labelwidth=1in,align=parleft,itemsep=0.1\baselineskip,leftmargin=!}
         
        \section*{List of Abbreviations}
        \addcontentsline{toc}{section}{List of Abbreviations}
        %\chaptermark{List of Abbreviations}
       
        
        \begin{abbrv}
        \item[BRAP]			Business Reforms Action Plan
        \item[CPCB]			Central Pollution Control Board
        \item[CRA]			Computerised Risk Assessment
        \item[CTE]				Consent to Establish
        \item[CTO]			Consent to Operate
        \item[DIPP]			Department of Industrial Policy and Promotion
        \item[DPCC]			Delhi Pollution Control Committee
        \item[DG]				Diesel Generator
        \item[EE]				Environmental Engineer
        \item[SEE]		                 Senior Environmental Engineer
        \item[SOP]			Standard Operating Procedure
        \item[SPCB]			State Pollution Control Board
        \item[VPI]				Visit Priority Index 
          
        
         
        \end{abbrv}
        
                    
                    %EXECUTIVE SUMMARY
                    \newpage
                    \section*{Executive Summary}
                    \addcontentsline{toc}{section}{Executive Summary}
                    The 2017 Business Reform Action Plan (BRAP) of the Department of Industrial Policy and Promotion (DIPP) of the Ministry of Commerce and Industry presented 405 recommendations for states to implement for improving the business climate, including six reforms to the way regulatory inspections are carried out. \\
                  
                  A rationalised, consistent and effective inspection regime is the starting point for regulatory enforcement. Unfortunately, inspections have long been infamous in India as the breeding ground for corruption and the extortion of money by government officials based on discretionary powers assigned to them (PHD Research Bureau 2015). With the focus on ease of doing business, reforms to end the inspector raj (Nanda 2014) have been front and centre on the agenda. \\
                  
                  The six inspection enabler reforms suggested by the DIPP can potentially change the way inspections are conducted today, particularly in the area of environmental compliance by the State Pollution Control Boards (SPCB). A risk-based approach to target inspections, transparency measures and process checklists, introduced in this round of reforms, are some of the best practices implemented globally for inspection reform (Blanc 2012, p. 3). By introducing a computerised risk assessment system, the SPCBs could reduce the discretionary power Environmental Engineers (EEs) (loosely called inspectors) wield over enterprises. EEs are now required to upload the inspection reports within 48 hours of the inspection, and reports can no longer be modified after they have been uploaded. \\
                  
                 The Delhi Pollution Control Committee (DPCC) is the SPCB responsible for the enforcement of environmental standards for industries in Delhi stipulated under the Air and Water Acts and Hazardous Waste Management Rules (Delhi Pollution Control Committee 2016b). \\
                 
                 In this paper, we assess the quality of implementation of different inspection reforms implemented by the DPCC in 2017 as part of the effort to improve the ease of doing business in Delhi. The inspections norms of DPCC have been the primary vehicle through which it ensures compliance and are also how businesses experience regulations and interact with the agency (DPCC 2016b). \\
                 
                 To study the extent to which the DPCC, which is the SPCB of Delhi, has implemented these reform suggestions and assess how the reforms have affected the functioning of DPCC, we interviewed Senior Environmental Engineers (SEEs) and businesses. In the course of our investigation, we also studied the Standard Operating Procedure (SOP) that EEs are required to follow to ensure the consistent implementation of regulations. We have not analysed the content and quality of SOPs or the BRAP reforms and have restricted ourselves to studying the implementation of the SOPs and recommended reforms. \\
                 
                 We find that while the DPCC has implemented some of the reforms it has claimed to, some are executed only partially, failing to meet the intended objectives. For example, the DPCC has published the inspection checklist on its website and now provides a unique login identification (ID) to enterprises to access the inspection reports. However, the DPCC does not yet use risk assessment software to identify enterprises for inspection—a key reform that it has claimed to have implemented. Even now, a committee manually identifies enterprises that are to be inspected. The DPCC claims that it cannot implement this system due to a shortage of personnel, as only 60\% of its sanctioned posts are currently filled. This is counterintuitive, as technology should ease some of the administrative workloads and help officials focus on primary tasks such as site inspection. \\
                 
                 Additionally, we find that EEs only partially follow the SOPs, raising concerns of procedural consistency across inspections. Enterprises are not always asked to sign the inspection report, and a formal notice detailing corrective measures is not always sent to enterprises. Not all enterprises have access to their inspection reports within 24 hours, and not all enterprises are notified of their risk category. Regardless of how well an SOP is designed, if it is not followed, it is likely to fail its intended purpose. \\
                 
                  The DIPP has so far primarily relied on self-reported evidence by states. In 2017, it introduced a system to include business feedback. Although business feedback is key to reform, it has its shortcomings. For instance, a business survey may not be able to judge the efficacy of changes to the back-office functioning of a government department. The current state of affairs calls for a close evaluation of the implementation of reforms, as reliance on self-reported evidence by state governments or a survey of business enterprises may paint a false picture. \\
                  
                  %INTRODUCTION
                    \newpage
                    \section{Introduction}
                    
                 The DPCC enforces national pollution standards in Delhi and is responsible for the ‘entire environmental status of the State’ (National Green Tribunal 2017). \\
                 
                 Although there are over 200 regulations governing environmental protection, the Water (Prevention and Control of Pollution) Act, 1974, and the Air (Prevention and Control of Pollution) Act of 1981 are the two essential acts that empower SPCBs to regulate the emissions and effluents discharged by any industry (OECD 2006). The DPCC also grants Consent to Establish (CTE) and Consent to Operate (CTO) to industries to establish and operate under these two acts. \\
                 
                 The quality of a regulatory regime and its effectiveness in achieving the regulatory goals depends as much on the way regulations are implemented and enforced as on the design of regulations. Unfortunately, the enforcement and implementation of regulations are not evaluated as often as regulation design, creating an informational lacuna (OECD 2014). \\
                 
                 One of the critical ways of carrying out regulatory action is through inspections. While the regulatory regime sets the rules for compliance, inspectors—who are at the frontlines of enforcement—generally have some autonomy and discretion in the way they enforce regulations (May and Wood 2003, p. 117). For most businesses, inspections are the primary form through which regulations are experienced and are a particularly important concern because they are recurring in nature (Blanc 2012, p. 7). \\
                 
                 Inspection is the primary tool used by the DPCC to monitor and enforce compliance environment standards. As per the Office Order of DPCC dated 26 July 2016, it conducts inspections to meet three objectives: first, to assess pollution potential; second, to evaluate compliance with standards stipulated for industries under the environment acts; and third, to guide industries to improve (DPCC 2016b). \\
                 
                 \subsection{Regulatory Reforms Introduced in 2017}
                 
                 Regulatory reforms are generally motivated by the need to ease the regulatory burden on businesses, to improve compliance and to improve government efficiency. The DIPP, of the Ministry of Commerce and Industry, conceived the BRAP in 2014, primarily to improve the ease of doing business and improve government efficiency. The implementation of reforms is evaluated by the DIPP and the World Bank periodically based on self-reported evidence. As of June 2018, Delhi had implemented 33.9\% of the reforms recommended under the BRAP and was ranked 23rd out of 36 states. \\
                 
                 Besides self-reported evidence, the DIPP and the World Bank also verified implementation through business surveys in 2017. However, there are two challenges in the use of a business survey to evaluate implementation. First, businesses may not be aware of governance changes introduced within government departments, and second, it sought to verify only selected reforms. Some improvements that have no direct bearing on business activity may remain unverified. \\
                 
                 Realising the gaps in evaluating the status of reforms in Delhi, we set out to study the enforcement of one part of the 2017 plan: inspections reforms. These inspections reforms apply to state government departments, such as the Labour Department, SPCBs and the Forest Department. In our study, we have only explored the reforms suggested and implemented by the SPCB in Delhi, that is, the DPCC. \\
                 
                 To streamline the inspection process of the SPCBs, BRAP 2017 recommended six reforms each under the Air and Water Acts. The six reforms under the Air Act are identical to those under the Water Act. \\
                 
                 The six distinct reforms (which are the same for the Air Act and the Water Act) include: \\
                 
                 \begin{enumerate}
                 \item Design and implement a system for identifying enterprises that need to be inspected based on a computerised risk assessment \textit{(Recommendations 165 and 171)}.
                 \item Publish a well-defined inspection procedure and checklist on the department’s website \textit{(Recommendations 164 and 170)}.
                 \item Allow enterprises to view and download submitted inspection reports for at least the past 2 years \textit{(Recommendations 167 and 173)}.
                 \item Mandate the online submission of inspection reports within 48 hours of the inspection to the DPCC \textit{(Recommendations 166 and 172)}.
                 \item Design and implement a system for the computerised allocation of inspectors \textit{(Recommendations 168 and 173)}.
                 \item Mandate that the same inspector will not inspect the same enterprise twice consecutively \textit{(Recommendations 169 and 175)} (Department of Industrial Policy and Promotion 2017a).
                 \end{enumerate}
                 
                 Delhi has provided evidence on the DIPP website on the implementation of a total of three of these six reforms under each Act. \\
                 
                 The first reform, identification of enterprises for inspection through a computerised risk assessment (CRA) system, reduces not only the burden of administration (DIPP 2017d) but also the scope for bias in the selection of enterprises and, therefore, the scope for inspector raj (Nanda 2014). \\
                 
                 Computerisation has a couple of benefits: the integration of inspection processes into one system and the elimination of overlapping inspections and the repetition of work (PWC 2017). Findings made by an inspectorate can also be relevant to other agencies. This data can be used to have a current assessment of the risk level of each business, without spending additional resources (OECD, 2014). The identification of enterprises for inspection through a computerised system allows for reduced bias, limits human errors and increases transparency. \\
                 
                 Besides the CRA system, three reforms aim to make information on the inspection procedure publicly available and enable enterprises to view inspection reports online, enhancing transparency. Publicly available checklists allow enterprises to be aware of the expectations of them and bring consistency in the enforcement of norms (Blanc 2012, p. 79). Improving access to information is critical to the quality of government service, empowers citizens and ensures greater accountability from government officials. In fact, easy access to regulatory information is associated with improved governance and reduced corruption (Geginat and Saltane 2016, p. 2). \\
                 
                 The paper focuses on the pollution inspection reforms implemented by the DPCC under BRAP 2017. The \hyperref[sec:1]{first section} of the paper examines the implementation status of the CRA system in Delhi, as it is the most substantive recommendation made by the DIPP. The \hyperref[sec:2]{second section} of the paper examines the implementation of other process improvement reforms and presents a preliminary assessment of the use of SOPs for inspection in the DPCC. Study findings are based on structured interviews with government officials at the DIPP, SEEs at the DPCC and enterprises. \\ %hyperref purpose? shouldn't it include [hidelinks] in the preamble?
                 
                 \section{Assessing the Implementation Status of Computerised Risk Assessment for Environment Inspections}\label{sec:1} 
                 
                 The DIPP recommended that the DPCC should design and implement a computerised system to identify enterprises for inspection based on a risk assessment. Risk is the probability and scale of the impact of an occurrence. Risk assessment has two aspects: identifying risk and taking measures to control or eliminate it (Stoneburner, Goguen and Feringa 2002). In the context of environmental protection, risk assessment refers to evaluating potential harm to humans, flora and fauna (Environmental Protection Agency, n.d.). The assessment of potential harm helps to identify the people and regions most susceptible to the risk, to prioritise hazards and to determine whether control measures are required for a particular hazard. \\
                 
                 Most countries have developed their own method of assessing environmental risk. The DIPP in the BRAP reforms (Recommendations 165 and 171) prescribed the implementation of the CRA system to identify enterprises for inspection. \\ %should recs be italicised to maintain consistency?
                 
                 According to DIPP officials, the CRA system involves using a computer software to assess risk. The results of the CRA system can be used to identify enterprises eligible for inspection. The DIPP recommends that this system of identifying enterprises must be computerised. The computerisation of the risk assessment process reduces the EEs’ role in assigning risk categories to enterprises when they apply for CTE and CTO and in deciding if, when and by whom they have to be inspected. It results in two benefits: elimination of human errors and bias (and scope for rent seeking) during  risk assessment and freeing up the risk assessor’s time. Moreover, computerisation allows for easy and transparent access to the results of assessment and inspection. \\
                 
                 In Sections 3.1. to 3.3., we evaluate the degree to which the computerisation of risk assessment has been introduced. Section 3.1. describes the risk assessment process currently in place, and section 3.2. discusses how the DPCC currently identifies enterprises for inspection post the risk assessment. Section 3.3. examines risk assessment practices in other states. \\ %how to label 3.1 and 3.2?
                 
                 \subsection{Risk Assessment Currently Practised by the Delhi Pollution Control Committee}\label{sec:1.1} %is this label correct?
                 
                 The DPCC uses the risk assessment method devised by the Central Pollution Control Board (CPCB) in 2016 to assess the risk of industries. It categorises industries as Red, Orange, Green or White based on their potential to pollute air and water. \\
                 
                 The pollution potential index takes into account the emissions of an enterprise (air pollutants), effluents (water pollutants), hazardous waste generated and consumption of resources. The pollution potential index is calculated based on the number and quantity of pollutants typically released by each industry. The current system assumes that each enterprise within an industry type would discharge similar amounts of effluents, emissions and waste and consume similar types and quantities of resources. The score for any industry ranges from 0 to 100, where a higher score indicates a higher pollution potential. \\
                 
                 % Table generated by Excel2LaTeX from sheet 'Sheet1'
% Table generated by Excel2LaTeX from sheet 'Sheet1'
\begin{table}[htbp]
  \centering
  \caption{CPCB Risk Categories}
    \begin{tabular}{rrrlrrrrrrr}
  \multicolumn{1}{l}{\bfseries{Risk Category}} & \multicolumn{1}{l}{\bfseries{Pollution Index Score}} & \multicolumn{1}{l}{\bfseries{Validity Period of CTE/CTO}} & {\bfseries{Examples of Activities}} &       &       &       &       &       &       &  \\ \hline
    \multicolumn{1}{l}{Red} & \multicolumn{1}{l}{60-100} & \multicolumn{1}{l}{5 years} & - Healthcare establishments &       &       &       &       &       &       &  \\
          &       &       & - Automobile manufacturing &       &       &       &       &       &       &  \\
          &       &       & - Slaughterhouses &       &       &       &       &       &       &  \\
    \multicolumn{1}{l}{Orange} & \multicolumn{1}{l}{41-59} & \multicolumn{1}{l}{10 years} & - Bakery and confectionary units with a production capacity of more than 1 tonne per day with an oven or furnace &       &       &       &       &       &       &  \\
          &       &       & - Hotels with less than three stars or hotels with more than 20, but less than 100 beds &       &       &       &       &       &       &  \\
          &       &       & -Food and food processing, including fruit and vegetable processing &       &       &       &       &       &       &  \\
    \multicolumn{1}{l}{Green} & \multicolumn{1}{l}{21-40} & \multicolumn{1}{l}{15 years} & - Bakery and confectionary units with a production capacity of less than 1 tonne per day with a gas or electric oven &       &       &       &       &       &       &  \\
          &       &       & - Hotels with up to 20 rooms and without boilers &       &       &       &       &       &       &  \\
          &       &       & -Digital printing on polyvinyl chloride clothes &       &       &       &       &       &       &  \\
    \multicolumn{1}{l}{White} & \multicolumn{1}{l}{0-20} & \multicolumn{1}{l}{CTE/CTO not required} & - Repairing electric motors and generators using a dry mechanical process &       &       &       &       &       &       &  \\
          &       &       & -Blending and packing of tea &       &       &       &       &       &       &  \\
          &       &       & -Packing of powdered milk &       &       &       &       &       &       &  \                \end{tabular}%
  \label{tab:addlabel}%
\end{table}%
                 
                 In addition to the risk categories assigned by the CPCB, DPCC uses another set of risk categories for industries—I, II(a) and II(b)—to determine the members of the committee that will issue CTE/CTO to each industry.
                 
  % Table generated by Excel2LaTeX from sheet 'Sheet1'
\begin{table}[htbp]
  \centering
  \caption{Add caption}
    \begin{tabular}{rrlr}
    \multicolumn{1}{l}{Risk Category} & \multicolumn{1}{p{10.415em}}{Requirement of Pollution Control Devices} & \multicolumn{1}{p{12.585em}}{Members of the Committee That Issue CTE/ CTO} & \multicolumn{1}{l}{Decision-Making Time Period} \\
    \multicolumn{1}{l}{I} & \multicolumn{1}{p{10.415em}}{Does not require the installation of pollution control devices} & \multicolumn{1}{p{12.585em}}{SEEs of the concerned cell} & \multicolumn{1}{p{12em}}{Decision to issue CTE/CTO to be made within 7 days of receipt of the application} \\
    \multicolumn{1}{r}{\multirow{6}[0]{*}{II(a)}} & \multicolumn{1}{r}{\multirow{6}[0]{*}{Requires the installation of pollution control devices, such as the sewage or water treatment plants of Delhi Jal Board, common effluent treatment plants, power plants or municipal solid waste plants}} & \multicolumn{1}{l}{\multirow{6}[0]{*}{Chairman
Member Secretary
Two Engineering Professors
Director, Department of Environment 
SEEs of the concerned cell
EEs of the concerned cell
}} & \multicolumn{1}{l}{\multirow{6}[0]{*}{Decision to issue CTE/CTO to be made within 30 days of receipt of the application}} \\
          &       &       &  \\
          &       &       &  \\
          &       &       &  \\
          &       &       &  \\
          &       &       &  \\
    \multicolumn{1}{r}{\multirow{3}[0]{*}{II(b)}} & \multicolumn{1}{r}{\multirow{3}[0]{*}{Requires the installation of pollution control devices not listed under category II(a)}} & \multicolumn{1}{l}{\multirow{3}[0]{*}{Member Secretary
SEEs of concerned cell
EEs of concerned cell}} & \multicolumn{1}{r}{\multirow{3}[0]{*}{Decision to issue CTE/CTO to be made within 30 days of receipt of the application}} \\
          &       &       &  \\
          &       &       &  \\
    \end{tabular}%
  \label{tab:addlabel}%
\end{table}%




                 
             




\end{document}